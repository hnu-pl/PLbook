\printbibliography[title=참고문헌]

\AppendixTitleToToc
\AttachAppendixTitleToSecnum

\appendix
\appendixpage*

%\chapterstyle{appendixdefault}
%\pagestyle{hangul}

\section{과제 형식 예시}
\input{HWsample}

\newpage

\section{감사}
강의 노트를 교재의 형태로 이렇게 출간할 수 있게 된 데는 제가 컴퓨터과학을 전공하면서
인연을 맺게 된 은사님들과 정보과학회 프로그래밍언어연구회의 모든 분들에게 마땅히 감사한 마음이 있습니다.
하지만 여기에서 감사의 말씀을 표하기에는 너무 길어지는 관계로,
우선 이 책의 출간 과정에 직접적인 도움을 주신 출판사 관계자 분들,
수업 중에 오탈자를 보고해 준 수강생들, 그리고 프로그래밍언어론 과목에서
대학원생 조교로 수업 운영을 도우며 출판사를 섭외하는 등 다방면으로
신경써 준 제자인 손범준에게 이 지면을 빌어 간단히나마 감사의 기록을 남깁니다.

\subsection{초판 원고 작업본 오탈자 보고}
\subsubsection{2022년 1학기 프로그래밍언어론 수강생}
구준한,
권준호,
권혁준,
김동하,
김상운,
김이레,
김현,
백성욱,
백창현,
손현승,
신현수,
이승민,
이상빈,
장주안,
임영준,
장태영,
전석원,
정윤정,
정재훈,
황인규.

\begin{comment}
\newpage

\section{다음 판에 추가할지 고려중인 주제}
\subsection{Control}
Continuation-Passing Style,
Delimited Continuations,
Coroutines, Exceptions, Async-Await,
Algebraic Effects,
Functor/Applicative/Monad/Monoid/...

\url{https://www.microsoft.com/en-us/research/wp-content/uploads/2016/08/algeff-tr-2016-v2.pdf}

\subsection{Staged Computation}
Interpreter vs. Compiler, Futamura Projections, Partial Evaluation

\end{comment}


